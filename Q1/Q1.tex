%\documentclass[class=article, crop=false]{standalone}
\documentclass{article}

\usepackage{amssymb,amsfonts,amsthm,amsmath}
\usepackage{enumitem}
\usepackage{hyperref,xcolor}
\usepackage{mathtools}
\usepackage{listings}
\usepackage{amsmath}
\newcommand{\comb}[2]{{}_{#1}\mathrm{C}_{#2}}
\hypersetup{
    colorlinks,
    urlcolor={black}	%black!50!blue
}
\newcommand{\solution}

\begin{document}
\newcommand{\BEQA}{\begin{eqnarray}}
\newcommand{\EEQA}{\end{eqnarray}}
\newcommand{\define}{\stackrel{\triangle}{=}}
\bibliographystyle{IEEEtran}
\raggedbottom
\setlength{\parindent}{0pt}
\providecommand{\mbf}{\mathbf}
\providecommand{\pr}[1]{\ensuremath{\Pr\left(#1\right)}}
\providecommand{\qfunc}[1]{\ensuremath{Q\left(#1\right)}}
\providecommand{\sbrak}[1]{\ensuremath{{}\left[#1\right]}}
\providecommand{\lsbrak}[1]{\ensuremath{{}\left[#1\right.}}
\providecommand{\rsbrak}[1]{\ensuremath{{}\left.#1\right]}}
\providecommand{\brak}[1]{\ensuremath{\left(#1\right)}}
\providecommand{\lbrak}[1]{\ensuremath{\left(#1\right.}}
\providecommand{\rbrak}[1]{\ensuremath{\left.#1\right)}}
\providecommand{\cbrak}[1]{\ensuremath{\left\{#1\right\}}}
\providecommand{\lcbrak}[1]{\ensuremath{\left\{#1\right.}}
\providecommand{\rcbrak}[1]{\ensuremath{\left.#1\right\}}}
\theoremstyle{remark}
\title{Probability Assignment-1}
\author{\Large Thoutu Rahul Raj - FWC22002}
\date{}

\maketitle

%\begin{enumerate}[label=16.\arabic{enumi}.\arabic{enumii}]%,ref=\thesection.\theenumi.\theenumi]
\numberwithin{equation}{enumi}
\setcounter{enumi}{3}
\setcounter{enumii}{6}
\section*{\large Problem statement}
A game consists of tossing a one rupee coin 3 times and noting its outcome each time. Hanif wins if all the tosses give the same result i.e., three heads or three tails, and loses otherwise. Calculate the probability that Hanif will lose the game.\\
%\includegraphics[scale=0.5]{fig.pdf} 
\solution
Let $X_i\in \cbrak{0,1}, i = 1, 2, 3$ represent a coin toss, or, the Bernoulli random variable.  Then the outcome of the game is
\begin{align}
X = X_1+X_2+X_3	
\end{align}
If 
\begin{align}
\pr{X_i = 1} &= p,\\
\pr{X = k} &= \comb{n}{k}p^k\brak{1-p}^k, \quad k = 0,\dots, n
\end{align}
$X$ is known as a Binomial random variable.  For the given problem, $n = 3, p = \frac{1}{2}$ and the probability of a win is 
\begin{align}
\pr{X = 3} + \pr{X = 0} &= \frac{1}{8}+\frac{1}{8}
\\
&= \frac{1}{4}
\end{align}
The loss probability is then
\begin{align}
1-\frac{1}{4} = \frac{3}{4}
\end{align}

The python code for the distribution of data,\\
\noindent\fbox{%
    \parbox{\linewidth}{%
   	\url{https://github.com/ahilan22/fwc-2/tree/main/probability/assignment/codes/Q1.py} 
   	   }
   	 }

\end{document}
