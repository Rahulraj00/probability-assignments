\documentclass{article}

\usepackage{amssymb,amsfonts,amsthm,amsmath}
\usepackage{enumitem}
\usepackage{hyperref,xcolor}
\usepackage{mathtools}
\usepackage{listings}
\usepackage{amsmath}
\newcommand{\comb}[2]{{}_{#1}\mathrm{C}_{#2}}
\hypersetup{
    colorlinks,
    urlcolor={black}	%black!50!blue
}
\newcommand{\solution}

\begin{document}
\newcommand{\BEQA}{\begin{eqnarray}}
\newcommand{\EEQA}{\end{eqnarray}}
\newcommand{\define}{\stackrel{\triangle}{=}}
\bibliographystyle{IEEEtran}
\raggedbottom
\setlength{\parindent}{0pt}
\providecommand{\mbf}{\mathbf}
\providecommand{\pr}[1]{\ensuremath{\Pr\left(#1\right)}}
\providecommand{\qfunc}[1]{\ensuremath{Q\left(#1\right)}}
\providecommand{\sbrak}[1]{\ensuremath{{}\left[#1\right]}}
\providecommand{\lsbrak}[1]{\ensuremath{{}\left[#1\right.}}
\providecommand{\rsbrak}[1]{\ensuremath{{}\left.#1\right]}}
\providecommand{\brak}[1]{\ensuremath{\left(#1\right)}}
\providecommand{\lbrak}[1]{\ensuremath{\left(#1\right.}}
\providecommand{\rbrak}[1]{\ensuremath{\left.#1\right)}}
\providecommand{\cbrak}[1]{\ensuremath{\left\{#1\right\}}}
\providecommand{\lcbrak}[1]{\ensuremath{\left\{#1\right.}}
\providecommand{\rcbrak}[1]{\ensuremath{\left.#1\right\}}}
\theoremstyle{remark}
\title{Probability Assignment-1}
\author{\Large Thoutu Rahul Raj - FWC22002}
\date{}
\maketitle
%\begin{enumerate}[label=16.\arabic{enumi}.\arabic{enumii}]%,ref=\thesection.\theenumi.\theenumi]
\numberwithin{equation}{enumi}
\setcounter{enumi}{3}
\setcounter{enumii}{6}
\section*{\large Problem statement}
An urn contains 5 red and 5 black balls. A ball is drawn at random, its colour is noted and is returned to the urn. Moreover, 2 additional balls of the colour drawn are put in the urn and then a ball is drawn at random. What is the probability that the second ball is red?
\section*{\large Solution}
Let $X \in {0,1}$ where 0 represents black.  Let $X_1$ represent the event representing drawing the first ball. $X_2$ represent the event of drawing the second ball.
Then  probability of the second ball being red is
\begin{multline}
\pr{X_2 = 1} 
\\
= \pr{X_2 = 1,X_1 = 1}+\pr{X_2 = 1,X_1 = 0}
\\
=\pr{X_2 = 1|X_1 = 1}\pr{X_1 = 1}
\\
+\pr{X_2 = 1|X_1 = 0}\pr{X_1 = 0}
\end{multline}
From the given information,
\begin{align}
\pr{X_1 = 0} = \pr{X_1 = 1}
=\frac{5}{10}=\frac{1}{2}.
\end{align}
Also, 
\begin{align}
\pr{X_2 = 1|X_1 = 0} &= \frac{5}{5+2+7} = \frac{5}{12}
\\
\pr{X_2 = 1|X_1 = 1} &= \frac{5+2}{5+2+5} = \frac{7}{12}
\end{align}
Thus, 
\begin{align}
\pr{X_2 = 1} = \frac{7}{12}\times \frac{1}{2} 
+ \frac{5}{12}\times \frac{1}{2}
=\frac{1}{2}
\end{align}
\begin{lstlisting}
https://github.com/Rahulraj00/probability-assignments/tree/main/Q3/codes/Q3.py
\end{lstlisting}
\end{document}
